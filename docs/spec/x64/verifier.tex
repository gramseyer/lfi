\specchapter{LFI-x64 Verifier}

% Don't forget to undefine at the bottom
\providecommand{\rbase}{\code{\%r14}\xspace}
\providecommand{\rsp}{\code{\%rsp}\xspace}
\providecommand{\esp}{\code{\%esp}\xspace}
\providecommand{\gs}{\code{\%gs}\xspace}
\providecommand{\rX}{\code{\%rX}\xspace}
\providecommand{\rY}{\code{\%rY}\xspace}
\providecommand{\eX}{\code{\%eX}\xspace}
\providecommand{\eY}{\code{\%eY}\xspace}

\providecommand{\bundlesize}{32\xspace}
\providecommand{\bundlemask}{0xffffffe0\xspace}

\informative{The LFI-x64 verifier is specified in terms of x64 machine code, expressed using GNU AT\&T assembly syntax.}

\specsection{Terminology}

\specitem
A \term{bundle} is a \bundlesize-byte region of memory, starting at a \bundlesize-byte aligned address.

\specitem
A \term{macroinstruction} consists of one or more instructions that must all reside contiguously in the same \term{bundle}.

\specitem
The notation \rX, \rY refers to any of the following registers:
\code{\%rsp},
\code{\%rax},
\code{\%rcx},
\code{\%rdx},
\code{\%rbx},
\code{\%rbp},
\code{\%rdi},
\code{\%rsi},
\code{\%r8},
\code{\%r9},
\code{\%r10},
\code{\%r11},
\code{\%r12},
\code{\%r13},
\code{\%r14},
\code{\%r15},
\code{\%gs}.

\specitem
The notation \eX, \eY refers to any of the following registers, corresponding to the lower 32 bits of \rX:
\code{\%esp},
\code{\%eax},
\code{\%ecx},
\code{\%edx},
\code{\%ebx},
\code{\%ebp},
\code{\%edi},
\code{\%esi},
\code{\%r8d},
\code{\%r9d},
\code{\%r10d},
\code{\%r11d},
\code{\%r12d},
\code{\%r13d},
\code{\%r14d},
\code{\%r15d}.

\specitem
Sometimes, \rX/\rY/\eX/\eY may be used when only a subset of the listed
registers are encodable. In this case only the encodable subset should be
considered. When \rX/\rY/\eX/\eY is used multiple times in the same listing, it
refers to the same register each time.

\specitem
The notation \code{opN} may refer to any valid register or memory operand.

\specitem
An instruction \textit{writes} to register \rX if it modifies the
64-bit value stored in \rX.

\specsection{Bundles}

\specitem
No instruction may cross a \bundlesize-byte-aligned address.

\specsection{Register Accesses}

\specitem
No instruction may write to register \rbase.

\specitem
No instruction except the following may write to register \rsp.

\begin{enumerate}
    \item
        \term{macroinstruction}: \\
        \code{mov \eX, \esp} \\
        \code{orq \rbase, \rsp}
    \item
        \term{macroinstruction}: \\
        \code{add \$N, \esp} \\
        \code{orq \rbase, \rsp}
    \item
        \term{macroinstruction}: \\
        \code{sub \$N, \esp} \\
        \code{orq \rbase, \rsp}
\end{enumerate}

\specsection{Control Flow}

\specitem
\label{specitem:x64_jmpq}
No instruction except the following may use \code{jmpq *}

\begin{enumerate}
    \item
        \term{macroinstruction}: \\
        \code{andl \$\bundlemask, \eX} \\
        \code{orq \rbase, \rX} \\
        \code{jmpq *\rX}
    \item
        \term{macroinstruction}: \\
        \code{leaq 1f(\%rip), \%r11} \\
        \code{jmpq *N(\rbase)} \\
        \code{1:} \\ \\
        Where \code{N \% 8 == 0} and \code{N < 256}.
\end{enumerate}

\specitem
No instruction except the following may use \code{callq *}

\begin{itemize}
    \item
        \term{macroinstruction}: \\
        \code{andl \$\bundlemask, \eX} \\
        \code{orq \rbase, \rX} \\
        \code{nop}\footnote{Zero or more \codefoot{nop} instructions may be placed within the \term{macroinstruction}.} \\
        \code{callq *\rX}
\end{itemize}

\informative{It is recommended to raise a warning or error if the \code{callq} does not terminate the macroinstruction, but this is not required for the verifier's correctness.}

\specitem
The \code{ret} instruction is not permitted.

\specsection{Memory Accesses}

\specitem
Unless otherwise permitted\footnote{The second \codefoot{jmpq} \term{macroinstruction} (\ref{specitem:x64_jmpq}), and \codefoot{leaq} (\ref{specitem:x64_leaq}) instructions.}, a memory operand must be one of the following forms:

\begin{enumerate}
    \item \code{\gs:N(\eX)}
    \item \code{\gs:N(\eX, \eY, M)}
    \item \code{\gs:N(, \eY, M)}
    \item \code{N(\rsp)}
\end{enumerate}

\specitem
\label{specitem:x64_leaq}
An \code{leaq} instruction may use any memory operand.

\specitem
No instruction except the following may use \code{stosq}

\begin{itemize}
    \item
        \term{macroinstruction}: \\
        \code{movl \%edi, \%edi} \\
        \code{leaq (\rbase, \%rdi), \%rdi} \\
        \code{stosq}
\end{itemize}

\specitem
No instruction except the following may use \code{movsq}

\begin{itemize}
    \item
        \term{macroinstruction}: \\
        \code{movl \%edi, \%edi} \\
        \code{leaq (\rbase, \%rdi), \%rdi} \\
        \code{movl \%esi, \%esi} \\
        \code{leaq (\rbase, \%rsi), \%rsi} \\
        \code{movsq}
\end{itemize}

\specitem
No instruction except the following may use \code{maskmovdqu}

\begin{itemize}
    \item
        \term{macroinstruction}: \\
        \code{movl \%edi, \%edi} \\
        \code{leaq (\rbase, \%rdi), \%rdi} \\
        \code{maskmovdqu opN, opM}
\end{itemize}

\specsection{Permitted Instructions}

\informative{The set of permitted instructions has not yet been strictly identified in this draft specification. Overall, we expect to permit all non-system x86-64 instructions from the base ISA, as well as those enabled by the following CPU features: CMOV, CX8, FPU, SSE, SSE2, CMPXCHG16B, POPCNT, SSE3, SSE4\_1, SSE4\_2, BMI1, BMI2, LZCNT. Future versions may enable more extensions.}

\let\rbase\undefined
\let\rsp\undefined
\let\esp\undefined
\let\gs\undefined
\let\rX\undefined
\let\rY\undefined
\let\eX\undefined
\let\eY\undefined
\let\bundlesize\undefined
\let\bundlemask\undefined
